%\begin{filecontents*}{example.eps}
%!PS-Adobe-3.0 EPSF-3.0
%%BoundingBox: 19 19 221 221
%%CreationDate: Mon Sep 29 1997
%%Creator: programmed by hand (JK)
%%EndComments

%\end{filecontents*}
%
%\documentclass{svjour3}                     % onecolumn (standard format)
%\documentclass[smallcondensed]{svjour3}     % onecolumn (ditto)
\documentclass[12pt,oneside,a4paper]{article}  

\usepackage{apacite}
\usepackage{appendix}
\usepackage{amsmath}
\usepackage{amsthm}

\usepackage{amssymb} % for approx greater than
\usepackage{caption}
\usepackage{placeins} % for \FloatBarrier
\usepackage{graphicx}
%\usepackage{subcaption}
\usepackage{longtable}
\usepackage{setspace}
\usepackage{booktabs}
\usepackage{tabularx}
\usepackage{xcolor,colortbl}
\usepackage{chngpage}
\usepackage{natbib}
\bibpunct{(}{)}{,}{a}{}{;} 
\usepackage{url}
\usepackage{nth}
\usepackage{authblk}
\usepackage[most]{tcolorbox}
\usepackage[normalem]{ulem}
\usepackage{amsfonts}

% columns for longtable
%\usepackage{arydshln} % Dashed lines in matrices

\usepackage[margin=1in]{geometry}
%\doublespacing % for review

% line numbers to make review easier
%\usepackage{lineno}
%\linenumbers

%\usepackage{soul}% for \st{}

%%%%%%%%%%%%%%%%%%%%%%%%%%%%%%%%%%%%%%%%%%%%%%%%%%%%%%%%%%%%%%%%%%%%%%%%%%%%%%
% for section 4 math environments
%\theoremstyle{definition}
%\newtheorem{definition}{Definition}[section]
%\newtheorem{theorem}{Theorem}[section]
%\newtheorem{proposition}{Proposition}[section]
%\newtheorem{corollary}{Corollary}[proposition]
%\newtheorem{remark}{Remark}[section]
%
%%%%%%%%%%%%%%%%%%%%%%%%%%%%%%%%%%%%%%%%%%%%%%%%%%%%%%%%%%%%%%%%%%%%%%%%%%%%%%
%\begin{filecontents*}{example.eps}
%!PS-Adobe-3.0 EPSF-3.0
%%BoundingBox: 19 19 221 221
%%CreationDate: Mon Sep 29 1997
%%Creator: programmed by hand (JK)
%%EndComments
%gsave
%newpath
%  20 20 moveto
%  20 220 lineto
%  220 220 lineto
%  220 20 lineto
%closepath
%2 setlinewidth
%gsave
%  .4 setgray fill
%grestore
%stroke
%grestore
%\end{filecontents*}
%\RequirePackage{fix-cm}

\newcommand\ackn[1]{%
  \begingroup
  \renewcommand\thefootnote{}\footnote{#1}%
  \addtocounter{footnote}{-1}%
  \endgroup
}

% Affiliations in small font size
%\renewcommand\Affilfont{\small}
\newcommand{\absdiv}[1]{%
  \par\addvspace{.5\baselineskip}% adjust to suit
  \noindent\textbf{#1}\quad\ignorespaces
}

%\defcitealias{HMD}{HMD 2016}

% junk for longtable caption
%\AtBeginEnvironment{longtable}{\linespread{1}\selectfont}
%\setlength{\LTcapwidth}{\linewidth}

% sort van Raalte properly
% #1: sorting key, #2: prefix for citation, #3: prefix for bibliography
%\DeclareRobustCommand{\VAN}[3]{#2} % set up for citation
%\newcommand{\tc}{\quad\quad\text{,}}
%\newcommand{\tp}{\quad\quad\text{.}}
%%%%%%%%%%%%%%%%%%%%%%%%%%%%%%%
\begin{document}


\title{A note on independant time measures}

%\author{Tim Riffe \and Neil Mehta \and Daniel Schneider \and Mikko Myrskyl\"a}
\author[1]{Tim Riffe\thanks{riffe@demogr.mpg.de}}
\author[2]{Joel Cohen}


\affil[1]{Max-Planck-Institute for Demographic Research}
\affil[2]{The Rockefeller University}


%\authorrunning{Short form of author list} % if too long for running head

\maketitle

\vspace{-2em}
\begin{abstract}
\absdiv{Background} There are countless Lexis-like relationships between
linearly dependant time measures, and these can be combined into higher-order
Lexis identities with dense linear dependencies. One such relationship is the demographic time identity between age, period, cohort, time to death,
length of life, and time of death. Certain subsets of time measures in this
and other higher order identities consist in time measures that are independant
of one another.
\absdiv{Objective} We aim to describe the relationship between
independent time measures and the identities within which they are nested, with
special attention to the demographic time identity and its three independant
time dyads. We aim to determine whether data structured on such time dyads might
be useful in demographic research.
\absdiv{Data and Methods} We illustrate concepts based on data from the Colonial
Qu\'{e}bec Population Register.
\absdiv{Results}
\absdiv{Conclusions}
\end{abstract}



% bibliography
%\bibliographystyle{spbasic}
%\bibliography{references}  
\end{document}

