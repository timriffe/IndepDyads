
%% BioMed_Central_Tex_Template_v1.06
%%                                      %
%  bmc_article.tex            ver: 1.06 %
%                                       %

%%IMPORTANT: do not delete the first line of this template
%%It must be present to enable the BMC Submission system to
%%recognise this template!!

%%%%%%%%%%%%%%%%%%%%%%%%%%%%%%%%%%%%%%%%%
%%                                     %%
%%  LaTeX template for BioMed Central  %%
%%     journal article submissions     %%
%%                                     %%
%%          <8 June 2012>              %%
%%                                     %%
%%                                     %%
%%%%%%%%%%%%%%%%%%%%%%%%%%%%%%%%%%%%%%%%%


%%%%%%%%%%%%%%%%%%%%%%%%%%%%%%%%%%%%%%%%%%%%%%%%%%%%%%%%%%%%%%%%%%%%%
%%                                                                 %%
%% For instructions on how to fill out this Tex template           %%
%% document please refer to Readme.html and the instructions for   %%
%% authors page on the biomed central website                      %%
%% http://www.biomedcentral.com/info/authors/                      %%
%%                                                                 %%
%% Please do not use \input{...} to include other tex files.       %%
%% Submit your LaTeX manuscript as one .tex document.              %%
%%                                                                 %%
%% All additional figures and files should be attached             %%
%% separately and not embedded in the \TeX\ document itself.       %%
%%                                                                 %%
%% BioMed Central currently use the MikTex distribution of         %%
%% TeX for Windows) of TeX and LaTeX.  This is available from      %%
%% http://www.miktex.org                                           %%
%%                                                                 %%
%%%%%%%%%%%%%%%%%%%%%%%%%%%%%%%%%%%%%%%%%%%%%%%%%%%%%%%%%%%%%%%%%%%%%

%%% additional documentclass options:
%  [doublespacing]
%  [linenumbers]   - put the line numbers on margins

%%% loading packages, author definitions

%\documentclass[twocolumn]{bmcart}% uncomment this for twocolumn layout and comment line below
\documentclass{bmcart}

%%% Load packages
%\usepackage{amsthm,amsmath}
%\RequirePackage{natbib}
%\RequirePackage[authoryear]{natbib}% uncomment this for author-year bibliography
%\RequirePackage{hyperref}
\usepackage[utf8]{inputenc} %unicode support
\usepackage{apacite}
\usepackage{appendix}
\usepackage{amsmath}
\usepackage{amsthm}
% for ASY interactive 3d figure
\usepackage[inline]{asymptote}
\usepackage{amssymb} % for approx greater than
%\usepackage{caption}
\usepackage{placeins} % for \FloatBarrier
\usepackage{graphicx}
\usepackage{subcaption}
\usepackage{longtable}
\usepackage{setspace}
\usepackage{booktabs}
\usepackage{tabularx}
\usepackage{xcolor,colortbl}
\usepackage{chngpage}
\usepackage{natbib}
%\bibpunct{(}{)}{,}{a}{}{;} 
\usepackage{url}
\usepackage{nth}
\usepackage{authblk}
\usepackage[most]{tcolorbox}
\usepackage[normalem]{ulem}
\usepackage{amsfonts}
\usepackage{censor}
\usepackage{etoolbox}
\AtBeginEnvironment{quote}{\singlespace\small}
\AtEndEnvironment{quote}{\endsinglespace}
%\usepackage[applemac]{inputenc} %applemac support if unicode package fails
%\usepackage[latin1]{inputenc} %UNIX support if unicode package fails


%%%%%%%%%%%%%%%%%%%%%%%%%%%%%%%%%%%%%%%%%%%%%%%%%
%%                                             %%
%%  If you wish to display your graphics for   %%
%%  your own use using includegraphic or       %%
%%  includegraphics, then comment out the      %%
%%  following two lines of code.               %%
%%  NB: These line *must* be included when     %%
%%  submitting to BMC.                         %%
%%  All figure files must be submitted as      %%
%%  separate graphics through the BMC          %%
%%  submission process, not included in the    %%
%%  submitted article.                         %%
%%                                             %%
%%%%%%%%%%%%%%%%%%%%%%%%%%%%%%%%%%%%%%%%%%%%%%%%%


% Toggle show graphics
%\def\includegraphic{}
%\def\includegraphics{}



%%% Put your definitions there:
\startlocaldefs
\endlocaldefs


%%% Begin ...
\begin{document}

%%% Start of article front matter
\begin{frontmatter}

\begin{fmbox}
\dochead{Research}

%%%%%%%%%%%%%%%%%%%%%%%%%%%%%%%%%%%%%%%%%%%%%%
%%                                          %%
%% Enter the title of your article here     %%
%%                                          %%
%%%%%%%%%%%%%%%%%%%%%%%%%%%%%%%%%%%%%%%%%%%%%%

\title{Independent time measures: Supplementary material 1}

%%%%%%%%%%%%%%%%%%%%%%%%%%%%%%%%%%%%%%%%%%%%%%
%%                                          %%
%% Enter the authors here                   %%
%%                                          %%
%% Specify information, if available,       %%
%% in the form:                             %%
%%   <key>={<id1>,<id2>}                    %%
%%   <key>=                                 %%
%% Comment or delete the keys which are     %%
%% not used. Repeat \author command as much %%
%% as required.                             %%
%%                                          %%
%%%%%%%%%%%%%%%%%%%%%%%%%%%%%%%%%%%%%%%%%%%%%%

\author[
   addressref={aff1},                   % id's of addresses, e.g. {aff1,aff2}
   corref={aff1},                       % id of corresponding address, if any
   noteref={n1},                        % id's of article notes, if any
   email={riffe@demogr.mpg.de}   % email address
]{\inits{TR}\fnm{Tim} \snm{Riffe}}
\author[
   addressref={aff2, aff3},
   email={cohen@mail.rockefeller.edu}
]{\inits{JEC}\fnm{Joel E.} \snm{Cohen}}

%%%%%%%%%%%%%%%%%%%%%%%%%%%%%%%%%%%%%%%%%%%%%%
%%                                          %%
%% Enter the authors' addresses here        %%
%%                                          %%
%% Repeat \address commands as much as      %%
%% required.                                %%
%%                                          %%
%%%%%%%%%%%%%%%%%%%%%%%%%%%%%%%%%%%%%%%%%%%%%%

\address[id=aff1]{%                           % unique id
  \orgname{Max-Planck-Institute for Demographic Research} % university, etc
 % \street{Konrad-Zuse-Str 1},                     %
 % \postcode{18057}                                % post or zip code
 % \city{Rostock},                              % city
 % \cny{DE}                                    % country
}
\address[id=aff2]{%
  \orgname{The Rockefeller University}
 % \street{D\"{u}sternbrooker Weg 20},
 % \postcode{24105}
 % \city{Kiel},
 % \cny{Germany}
}
\address[id=aff3]{%
  \orgname{Columbia University}
  %\street{D\"{u}sternbrooker Weg 20},
  %\postcode{24105}
  %\city{Kiel},
  %\cny{Germany}
}


\begin{artnotes}
%\note{Sample of title note}     % note to the article
\note[id=n1]{Equal contributor} % note, connected to author
\end{artnotes}

\end{fmbox}% comment this for two column layout

%\end{fmbox}% uncomment this for twcolumn layout

\end{frontmatter}

%%%%%%%%%%%%%%%%%%%%%%%%%%%%%%%%%%%%%%%%%%%%%%
%%                                          %%
%% The Main Body begins here                %%
%%                                          %%
%% Please refer to the instructions for     %%
%% authors on:                              %%
%% http://www.biomedcentral.com/info/authors%%
%% and include the section headings         %%
%% accordingly for your article type.       %%
%%                                          %%
%% See the Results and Discussion section   %%
%% for details on how to create sub-sections%%
%%                                          %%
%% use \cite{...} to cite references        %%
%%  \cite{koon} and                         %%
%%  \cite{oreg,khar,zvai,xjon,schn,pond}    %%
%%  \nocite{smith,marg,hunn,advi,koha,mouse}%%
%%                                          %%
%%%%%%%%%%%%%%%%%%%%%%%%%%%%%%%%%%%%%%%%%%%%%%

%%%%%%%%%%%%%%%%%%%%%%%%% start of article main body
% <put your article body there>

\section*{Interactive 3d diagrams}

This excerpt repeats Fig.~2 from the main manuscript, replacing the 2d renderings with 3d interactive diagrams. Spending some time rotating the diagrams may add intuition to the View Axes section, here repeated in full.

\subsection*{View axes}
\label{sec:viewaxes}
Imagine now the 3d projection of the graph from Fig.~\ref{fig:tet}, a regular tetrahedron. An angle of view directly at any of the four faces of the demographic time tetrahedron reveals the plane-perspective of a Lexis-like identity. In other words, the viewing axis is set as \emph{normal} to one of the faces of the tetrahedron. Interactive Fig.~\ref{fig:viewaxes} may help visualize the notion of viewing axes--- the view on a plane is orthogonal if an axis line is reduced to appear as a point in the centroid of the tetrahedron. The view axes of the four triad identities map to the four medians (axes of 3-fold symmetry) of the tetrahedron, depicted in Fig.~\ref{fig:depviewaxes} with cyan colored lines. Independent dyad planes are revealed when the viewing angle is such that the midpoints of opposite tetrahedral edges are aligned. There are three pairs of opposite edges (LP, TC, AD), and therefore three viewing angles of this kind, which map to the bimedians of the tetrahedron (axes of 2-fold symmetry), depicted in Figure~\ref{fig:indepviewaxes} with magenta lines.

\begin{figure}
\begin{subfigure}[t]{0.45\linewidth}
    \centering
    \begin{asy}
// DepViewAxes produced by rgl
settings.prc = true;
size(3inches, 3inches);
import graph3;
currentprojection = orthographic(0, -3.101144, 1.128724, up = (0, 0.3420201, 0.9396926));
defaultpen(fontsize(14));
ticklabel RGLstrings(real[] at, string[] label)
{
  return new string(real x) {
    int i = search(at, x);
    if (i < 0) return "";
    else return label[i];
  };
}

ticklabel RGLScale(real s)
{
  return new string(real x) {return format(s*x);};
}
currentlight = light(ambient=new pen[] {rgb(1,1,1)},
diffuse = new pen[] {rgb(1,1,1)},
specular = new pen[] {rgb(1,1,1)},
position = new triple[] {(0,0,1)},
viewport = true);
currentpen += linewidth(4);
currentpen = colorless(currentpen) + rgb(0.1921569, 0.5686275, 0.7882353);
draw((0.4985029, 0, -0.1762474)
--(-0.2492514, 0.4317162, -0.1762474)
);
label("P", position = (-0.002741766, 0.3181748, -0.1938721), align = (0,0));
currentpen = colorless(currentpen) + rgb(0.8235294, 0.7372549, 0.1764706);
draw((0.4985029, 0, -0.1762474)
--(-0.2492514, -0.4317162, -0.1762474)
);
label("C", position = (-0.002741766, -0.3181748, -0.1938721), align = (0,0));
currentpen = colorless(currentpen) + rgb(0.5333334, 0.1215686, 0.5764706);
draw((0.4985029, 0, -0.1762474)
--(0, 0, 0.5287422)
);
label("D", position = (0.1809565, 0, 0.3257052), align = (0,0));
currentpen = colorless(currentpen) + rgb(0.8235294, 0.2156863, 0.2156863);
draw((-0.2492514, 0.4317162, -0.1762474)
--(-0.2492514, -0.4317162, -0.1762474)
);
label("A", position = (-0.2741766, -0.1614619, -0.1938721), align = (0,0));
currentpen = colorless(currentpen) + rgb(0.3058824, 0.7882353, 0.2313726);
draw((-0.2492514, 0.4317162, -0.1762474)
--(0, 0, 0.5287422)
);
label("T", position = (-0.09047827, 0.156713, 0.3257052), align = (0,0));
currentpen = colorless(currentpen) + rgb(0.772549, 0.4588235, 0.1686275);
draw((-0.2492514, -0.4317162, -0.1762474)
--(0, 0, 0.5287422)
);
label("L", position = (-0.09047827, -0.156713, 0.3257052), align = (0,0));
currentpen += linewidth(2);
currentpen = colorless(currentpen) + rgb(0, 1, 1);
draw((0, 0, 0.6873648)
--(0, 0, -0.2291216)
);
currentpen = colorless(currentpen) + rgb(0, 0, 0);
label("APC", position = (0, 0, -0.2291216), align = (0,0));
currentpen = colorless(currentpen) + rgb(0, 1, 1);
draw((-0.3240269, -0.561231, -0.2291216)
--(0.108009, 0.187077, 0.07637387)
);
currentpen = colorless(currentpen) + rgb(0, 0, 0);
label("TPD", position = (0.108009, 0.187077, 0.07637387), align = (0,0));
currentpen = colorless(currentpen) + rgb(0, 1, 1);
draw((-0.3240269, 0.561231, -0.2291216)
--(0.108009, -0.187077, 0.07637387)
);
currentpen = colorless(currentpen) + rgb(0, 0, 0);
label("CDL", position = (0.108009, -0.187077, 0.07637387), align = (0,0));
currentpen = colorless(currentpen) + rgb(0, 1, 1);
draw((0.6480538, 0, -0.2291216)
--(-0.2160179, 0, 0.07637387)
);
currentpen = colorless(currentpen) + rgb(0, 0, 0);
label("TAL", position = (-0.2160179, 0, 0.07637387), align = (0,0));
currentlight.background = rgb(0.2980392, 0.2980392, 0.2980392);
currentlight.background = rgb(1, 1, 1);
currentlight.background = rgb(1, 1, 1);
\end{asy}

    \caption{The four view axes of the triad dependencies map to the medians of the tetrahedron. Rotate the tetrahedron so that a given cyan line is reduced to a point: The resulting view is then orthogonal to the outermost identity when flattened to 2d.}
    \label{fig:depviewaxes}
\end{subfigure}
~~
\begin{subfigure}[t]{0.45\linewidth}
    \begin{asy}
// IndepViewAxes produced by rgl
settings.prc = true;
size(3inches, 3inches);
import graph3;
currentprojection = orthographic(0, -2.640817, 0.9611788, up = (0, 0.3420201, 0.9396926));
defaultpen(fontsize(14));
ticklabel RGLstrings(real[] at, string[] label)
{
  return new string(real x) {
    int i = search(at, x);
    if (i < 0) return "";
    else return label[i];
  };
}

ticklabel RGLScale(real s)
{
  return new string(real x) {return format(s*x);};
}
currentlight = light(ambient=new pen[] {rgb(1,1,1)},
diffuse = new pen[] {rgb(1,1,1)},
specular = new pen[] {rgb(1,1,1)},
position = new triple[] {(0,0,1)},
viewport = true);
currentpen += linewidth(4);
currentpen = colorless(currentpen) + rgb(0.1921569, 0.5686275, 0.7882353);
draw((0.5820827, 0, -0.2057973)
--(-0.2910413, 0.5040984, -0.2057973)
);
label("P", position = (-0.003201455, 0.3715205, -0.226377), align = (0,0));
currentpen = colorless(currentpen) + rgb(0.8235294, 0.7372549, 0.1764706);
draw((0.5820827, 0, -0.2057973)
--(-0.2910413, -0.5040984, -0.2057973)
);
label("C", position = (-0.003201455, -0.3715205, -0.226377), align = (0,0));
currentpen = colorless(currentpen) + rgb(0.5333334, 0.1215686, 0.5764706);
draw((0.5820827, 0, -0.2057973)
--(0, 0, 0.6173919)
);
label("D", position = (0.211296, 0, 0.3803134), align = (0,0));
currentpen = colorless(currentpen) + rgb(0.8235294, 0.2156863, 0.2156863);
draw((-0.2910413, 0.5040984, -0.2057973)
--(-0.2910413, -0.5040984, -0.2057973)
);
label("A", position = (-0.3201455, -0.1885328, -0.226377), align = (0,0));
currentpen = colorless(currentpen) + rgb(0.3058824, 0.7882353, 0.2313726);
draw((-0.2910413, 0.5040984, -0.2057973)
--(0, 0, 0.6173919)
);
label("T", position = (-0.105648, 0.1829877, 0.3803134), align = (0,0));
currentpen = colorless(currentpen) + rgb(0.772549, 0.4588235, 0.1686275);
draw((-0.2910413, -0.5040984, -0.2057973)
--(0, 0, 0.6173919)
);
label("L", position = (-0.105648, -0.1829877, 0.3803134), align = (0,0));
currentpen += linewidth(2);
currentpen = colorless(currentpen) + rgb(1, 0, 1);
draw((-0.436562, 0, -0.308696)
--(0.436562, 0, 0.308696)
);
currentpen = colorless(currentpen) + rgb(0, 0, 0);
label("AD", position = (-0.4656661, 0, -0.3292757), align = (0,0));
currentpen = colorless(currentpen) + rgb(1, 0, 1);
draw((-0.218281, 0.3780738, 0.308696)
--(0.218281, -0.3780738, -0.308696)
);
currentpen = colorless(currentpen) + rgb(0, 0, 0);
label("TC", position = (-0.2328331, 0.4032787, 0.3292757), align = (0,0));
currentpen = colorless(currentpen) + rgb(1, 0, 1);
draw((-0.218281, -0.3780738, 0.308696)
--(0.218281, 0.3780738, -0.308696)
);
currentpen = colorless(currentpen) + rgb(0, 0, 0);
label("LP", position = (-0.2328331, -0.4032787, 0.3292757), align = (0,0));
currentlight.background = rgb(0.2980392, 0.2980392, 0.2980392);
currentlight.background = rgb(1, 1, 1);
currentlight.background = rgb(1, 1, 1);
\end{asy}

    \caption{The three view axes of the independent planes map to the bimedians of the tetrahedron. Rotate the tetrahedron so that a given magenta line is reduced to a point: The resulting view is then orthogonal to independent plane defined on the basis of the two perpendicular edges crossing in the middle of the figure.}
    \label{fig:indepviewaxes}       
\end{subfigure}
\caption{The seven principal view axes of the demographic time identity. Figures are interactive if viewed in a recent version of Adobe Reader. Click each figure to activate. A mouse wheel (or right click and drag) can be used to zoom. Left click and drag to rotate.}
\label{fig:viewaxes}
\end{figure}

Given a 3d volume of data structured by the demographic time identity, a cross-section orthogonal to a magenta line from Fig.~\ref{fig:indepviewaxes} is an independent plane. There is no reason not to do this for outcome measures, for example disease prevalence, in search of meaningful patterns. Further, when \emph{slicing} through such a complex timespace, we are agnostic about cut angles in general, but \emph{snapping} to axes defined on the basis of unit time measures leads to easier interpretation.

%%%%%%%%%%%%%%%%%%%%%%%%%%%%%%%%%%%%%%%%%%%%%%
%%                                          %%
%% Backmatter begins here                   %%
%%                                          %%
%%%%%%%%%%%%%%%%%%%%%%%%%%%%%%%%%%%%%%%%%%%%%%

\begin{backmatter}

%%%%%%%%%%%%%%%%%%%%%%%%%%%%%%%%%%%%%%%%%%%%%%%%%%%%%%%%%%%%%
%%                  The Bibliography                       %%
%%                                                         %%
%%  Bmc_mathpys.bst  will be used to                       %%
%%  create a .BBL file for submission.                     %%
%%  After submission of the .TEX file,                     %%
%%  you will be prompted to submit your .BBL file.         %%
%%                                                         %%
%%                                                         %%
%%  Note that the displayed Bibliography will not          %%
%%  necessarily be rendered by Latex exactly as specified  %%
%%  in the online Instructions for Authors.                %%
%%                                                         %%
%%%%%%%%%%%%%%%%%%%%%%%%%%%%%%%%%%%%%%%%%%%%%%%%%%%%%%%%%%%%%

% if your bibliography is in bibtex format, use those commands:
\bibliographystyle{spbasic} % Style BST file (bmc-mathphys, vancouver, spbasic).
\bibliography{references.bib}      % Bibliography file (usually '*.bib' )
% for author-year bibliography (bmc-mathphys or spbasic)

\end{backmatter}
\end{document}






